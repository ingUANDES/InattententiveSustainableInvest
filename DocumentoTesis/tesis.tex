\documentclass[12pt]{report}
\input{core/preambulo.tex}
\input{core/glosario.tex}

%%%%%%%%%%%%%%%%%%%%%%%%%%%%%%%%%%%%%%%%%%%%%%%%%%%%%%%%%%
\newcommand{\nombreautor}{Lucas Moyano}
\newcommand{\mes}{Mes}
\newcommand{\anio}{Año}
\newcommand{\titulo}{Sobre el rol de la inatención racional en inversiones con criterios ASG ante cambios regulatorios}
\newcommand{\nombreprofuno}{Sebastian Cea}
\newcommand{\nombreprofdos}{Nombre Profe 2}
\newcommand{\nombreproftres}{Nombre Profesor Invitado}
\newcommand{\codigo}{ING-IN-XXX/XX}

%%%%%%%%%%Aqui se escribe el resumen%%%%%%%%%%%%%%%%%%%%%

\newcommand{\dedicatoria}{
En esta sección se deben incluir, de ser necesario, los agradecimientos personales a familiares, colegas, etc. (Máximo 1 página)
 }

\newcommand{\resumen}{
En esta sección se debe incluir un resumen de los principales aspectos desarrollados en la tesis, justificación, metodología y principales resultados. (Máximo 1 página). 

\vspace{5cm}

{\color{blue}

El siguiente documento no puede ser la copia del escrito del o los art\'{\i}culos publicados (si corresponde).
Estos deben ser incluidos como Ap\'endices al final de la tesis.

\vspace{1cm}

Los m\'argenes son 2.5 cm superior e inferior, y 3 cm derecha e izquierda, en todo el documento.

\vspace{1cm}

Se debe mantener el formato tipo de letra Times New Roman, tama\~no 12 e interlineado 1.5 l\'{\i}neas.
}
}

\newcommand{\resumenIngles}{
En esta sección se debe incluir un resumen en Inglés de los principales aspectos desarrollados en la tesis, justificación, metodología y principales resultados. (Máximo 1 página).}


\newcommand{\agradecimientos}{
En esta sección, se deben incluir los agradecimientos en primer lugar a las fuentes de financiamiento de las tesis, si las hubiera y posteriormente a los académicos involucrados en el desarrollo de la investigación. (Máximo 1 página). 
}
\renewcommand{\baselinestretch}{1.5}

%%%%%%%%%%%%%%%%%%%%%%%%%%%%%%%%%%%%%%%%%%%%%%%%%%%%%%%%%%

% Por favor coloque todas las definiciones de simbolos a continuacion
% y no en los archivos de capitulos. Esto evitara la  existencia de
% multiples definiciones para las mismas palabras.

\newcommand{\mydef}
	{\stackrel{\mathrm{def}}{=}}
\newcommand{\e}
	{\hbox{\large{e}}}
\newcommand{\mi}
	{\hbox{\large{i}}}
\newcommand{\RR}{\mathbb{R}}


%%%%%%%%%%%%%%%%%%%%%%%%%%%%%%%%%%%%%%%%%%%%%%%%%%%%%%%%%%

\begin{document}
\label{start}

% Genera las paginas de titulo, copyright, resumen,
% agradecimientos, dedicatoria, indices,...

% No es necesario modificar

% Paginas de titulo, derechos de autor y otras mas.

% Jaime Cisternas, julio 2004, agosto 2010

% declaracion especifica de pdfLaTeX
% permite incluir figuras con \includegraphics[...]{figurename}
\DeclareGraphicsExtensions{.jpg,.pdf,.mps,.png}

%%%%%%%%%%%%%%%%%%%%%%%%%%%%%%%%%%%%%%%%%%%%%%%%%%%%%%%%%%%%%%%%

  \pagenumbering{roman}
% \setcounter{page}{1}

% Escribe la pagina de titulo, incluyendo el autor y la facultad

  \thispagestyle{empty}
  \textsc{
  \vspace*{0cm}
  \begin{center}
    \Large
     Universidad de los Andes \\
     Facultad de Ingeniería y Ciencias Aplicadas\\
  \end{center}
     \vspace{1cm}
  \begin{center}
     \includegraphics[angle=0,height=5cm]{logos/Logo-UANDES.png}
  \end{center}
     \vspace{1cm}
  \begin{center}
     \Large
     \titulo
  \end{center}
  \vspace{0.5cm}
  \begin{center}
    \Large
    \nombreautor
  \end{center}
  \vspace{0.5 cm}
  \begin{center}
    Tesis para optar al grado de \\
    MAGÍSTER/DOCTORADO EN CIENCIAS DE LA INGENIERÍA\\
    \vspace{1cm}
    Profesor Guía: \nombreprofuno \\
    Profesor Co-Guía: \nombreprofdos \\
    \vspace{0.5cm}
    \codigo \\
    \vspace{0.5cm}
    Santiago, \mes\ de \anio
  \end{center}
  }

%%%%%%%%%%%%%%%%%%%%%%%%%%%%%%%%%%%%%%%%%%

% Página en blanco
$\,$\\
\newpage
$\,$\\


% Escribe la segunda pagina de titulo para ser firmada por la comision

\cleardoublepage
\thispagestyle{empty}

%\begin{center}
%
%\vspace*{2cm}
%\parbox{10cm}{
%\noindent
%Certifico que he leído esta tesis y que en mi opinión
%su alcance y calidad son completamente adecuados como para ser considerada
%una tesis conducente al título de Ingeniero y grado de Magíster.
%\vspace{1cm}
%
%\hfill
%\begin{tabular}{c}
%\hspace{8cm} \\
%\hline
%\nombreprofuno \\
%(Profesor Guía)
%\end{tabular}
%
%\vspace*{1.5cm}
%
%\noindent
%Certifico que he leído esta tesis y que en mi opinión
%su alcance y calidad son completamente adecuados como para ser considerada
%una tesis conducente al título de Ingeniero y grado de Magíster.
%\vspace{0.75cm}
%
%\hfill
%\begin{tabular}{c}
%\hspace{8cm} \\
%\hline
%\nombreprofdos
%\end{tabular}
%
%\vspace*{1.5cm}
%
%\noindent
%Certifico que he leído esta tesis y que en mi opinión
%su alcance y calidad son completamente adecuados como para ser considerada
%una tesis conducente al título de Ingeniero y grado de Magíster.
%\vspace{0.75cm}
%
%\hfill
%\begin{tabular}{c}
%\hspace{8cm} \\
%\hline
%\nombreproftres
%\end{tabular}
%}
%
%\end{center}

%%%%%%%%%%%%%%%%%%%%%%%%%%%%%%%%%%%%%%%%%%

% Espaciado entre lineas
  \onehalfspacing
  %\doublespacing

% Hace pagina de derechos de autor

%  \cleardoublepage
%  \thispagestyle{empty}
%  %\vspace*{0in}
%  \begin{center}
%    \copyright\ \nombreautor\ \anio \\
%    Todos los derechos reservados.
%  \end{center}

%%%%%%%%%%%%%%%%%%%%%%%%%%%%%%%%%%%%%%%%%%%%%%%%%%%%%%%%%%%%%%%%%%%%%%


% Genera dedicatoria leyendo definicion de agradecimientos

  \cleardoublepage \addcontentsline{toc}{section}{Dedicatoria}
  \begin{center} \Large \textbf{Dedicatoria} \end{center}
  \dedicatoria

% Genera resumen leyendo definicion de resumen

  \cleardoublepage
  \refstepcounter{dummy} \addcontentsline{toc}{section}{Resumen}
  \begin{center} \Large \textbf{Resumen} \end{center}
  \resumen

% Genera resumen leyendo definicion de abstract
  \cleardoublepage
  \refstepcounter{dummy} \addcontentsline{toc}{section}{Abstract}
  \begin{center} \Large \textbf{Abstract} \end{center}
  \resumenIngles



% Genera agradecimientos leyendo definicion de agradecimientos

  \cleardoublepage \addcontentsline{toc}{section}{Agradecimientos}
  \begin{center} \Large \textbf{Agradecimientos} \end{center}
  \agradecimientos

% Produce indice general, listas de figuras y de tablas

  \cleardoublepage
  \tableofcontents
  \cleardoublepage
  \listoffigures
  \cleardoublepage
  \listoftables
  \cleardoublepage

  \normalsize
  \pagenumbering{arabic}

%%%%%%%%%%%%%%%%%%%%%%%%%%%%%%%%%%%%%%%%%%%%%%%%%%%%%%%%%%%


%%%%%%%%%%%%%%%%Ingreso de capitulos%%%%%%%%%%%%%%%%%%%%%%

% espaciado similar a Word 1.5
\spacing{1.4}

% cap1.tex

\chapter{Introducción}
\label{ch:intro} % la etiqueta para referencias

%{\color{blue} En esta sección de debe incluir el fundamento general de la motivación de la investigación desarrollada en la tesis. También se debe incluir la Hipótesis de trabajo, Objetivo General y específicos. (Máximo 5 páginas). Usar mismo tamaño, tipo de letra y espacio del este formato).}

%\vspace{0.5cm}

%The \Gls{latex} typesetting markup language is specially suitable 
%for documents that include \gls{maths}. \Glspl{formula} are 
%rendered properly an easily once one gets used to the commands.

%Given a set of numbers, there are elementary methods to compute 
%its \acrlong{gcd}, which is abbreviated \acrshort{gcd}. This 
%process is similar to that used for the \acrfull{lcm}.

%\vspace{0.5cm}
\section{Motivación}

Hoy en día se ha observado un creciente interés por parte de los inversores en los criterios Ambientales, Sociales y de Gobernanza (ASG o ESG por sus siglas en inglés).\footnote{En la motivación del Proyecto FONDECYT Drivers of Sustainable Finance se citan datos de que un tercio de los activos....} Esto se debe en gran medida al aumento en la conciencia de los inversionistas en temas los cuales hacen referencia a desafíos ambientales y sociales, tales como el cambio climático, la escasez de recursos naturales, distintas prácticas laborales y la ética empresarial. Asimismo, las empresas que logran seguir estas políticas consiguen traer a largo plazo un impacto significativo tanto en la sostenibilidad y rendimiento financiero. Por lo tanto, los tomadores de decisiones buscan elegir estas empresas que consiguen buenos resultados en estos tres factores, y de esta forma evitan compañías que basen sus ganancias en vicios (alcohol, tabaco y juegos). Este tipo de preocupaciones puede mejorar o facilitar condiciones para empresas que desarrollen productos y estrategias de inversión sostenible. Por ejemplo, empezando por fondos de inversión socialmente responsables, bonos verdes o hasta acciones de empresas enfocadas en energías renovables y buenas conductas empresariales. Pero las preguntas a plantearse son, ¿Realmente la gente les toma el real peso a estos criterios? ¿Realmente están atentos a los cambios de regulación que han tenido que implementar las empresas para que de esta forma guíen su decisión de si incorporan o no una empresa sobre la otra en el portafolio de inversión?\\

Teniendo en cuenta lo anterior, se busca aplicar un modelo de inversiones sostenibles el cual tenga un cambio en una regulación ASG propuesta por el gobierno, y ver la relevancia de la inatención a estos criterios en la valorización de los Retornos esperados en sus carteras de inversión. Es por esto que se busca por medio de los integrantes mencionados al inicio, una base de datos y un catastro donde se pueda analizar en que se están fijando hoy en día los inversores a la hora de querer invertir en sus portafolios, y si realmente, están al tanto de las regulaciones que cada vez toman más importancia en el ámbito financiero. Además, de querer crear un programa donde se reúna que información es relevante para los analistas al momento de querer crear su portafolio y si toman en cuenta estos factores ASG.

\vspace{0.5cm}

Ademas cada vez hay mas intereses en el impacto financiero que se esta teniendo con estos criterios ASG. Los estudios se centran en evaluar como los parámetros ASG pueden afectar el rendimiento y la rentabilidad de las inversiones. Se estan llevando a cabo investigaciones para determinar si las empresas con mejores practicas ASG tienen un mejor desempeño financiero a largo plazo y si los inversores que consideran estos criterios pueden obtener mejores resultados. En Chile, cada vez se están profundizando mas los criterios ASG, como se puede ver en Santander donde cuentan ya con fondos mutuos ASG, apoyando el medioambiente, la sociedad y las buenas prácticas. Con esto, podemos obtener distintas fuentes y bases de datos con tal de poder trabajar y así investigar de mejor forma en que se están basando en su toma de decisiones los inversionistas.

\vspace{0.5cm}

De acuerdo a la investigación realizada en el artículo de \citeA{pastor_sustainable_2021}, “Sustainable investing in equilibrium”, donde se modelan las inversiones realizadas en el mercado teniendo en cuenta los factores ASG (Ambiental, sociales, gobernanza) de las empresas, se demostró que, en equilibrio, los activos verdes (que tienen buenos factores ASG) superan el rendimiento esperado de los activos cafés (no con buenos factores ASG) cuando se producen impactos positivos en el factor ASG, el cual refleja cambios en los gustos de los consumidores por productos ecológicos y en los gustos de los inversores por tenencias verdes. También se hace mención de un modelo el cual de alguna forma calcula el gusto de la gente de querer añadir inversiones “verdes” a sus portafolios con tal de que ha futuro se generen retornos esperados futuros, con el plus de además de no dañar el medio ambiente, u otros factores ASG. 

\vspace{0.5cm}



\section{Hipótesis}

Esta tesis enuncia 2 hipótesis de trabajo descritas a continuación:

\begin{enumerate}
    \item Un cambio regulatorio en los estándares de reporte de métricas de cumplimiento relacionadas a ESG tiene efecto en la valorización de activos financieros. 
    \item Dada la complejidad de las métricas ESG, existe un componente en la valorización de activos relacionado a inatención racional.
\end{enumerate}

\section{Objetivos}

El trabajo se estructura en torno a un objetivo general que articula las hipótesis y N objetivos especificos que ...

\subsection{General} 

Evaluar el rol de la inatención racional de los tomadores de decisión ante un cambio en la regulación sobre reporte de métricas ESG en un modelo de inversión sostenible con varios activos en equilibrio.

\subsection{Específicos} 

Tomando en cuenta la revisión de literatura de \cite{pastor_sustainable_2021} y el modelo propuesto, estructuramos los siguientes objetivos específicos.

\begin{enumerate}
    \item Contextualizar en base a la literatura relacionada considerando los temas de: Compromiso de los inversionistas, subrendimiento de activos sostenibles, sobrerendimiento de activos sostenibles, relación entre inversión sostenible y lealtad cliente-empresa, impacto social de la inversión sostenible, y la caracterización de una frontera eficiente.
    \item Identificar resultados del modelo de \cite{pastor_sustainable_2021} que puedan ser validados con datos
    \item Incorporar inatención racional en la toma de decisiones del inversionista del modelo de \cite{pastor_sustainable_2021}
    \item Construir una base de datos para validar el objetivo específico 2 en dos periodos de tiempo (antes y después del cambio en la regulación)
    \item Analizar el grado de validación de las hipótesis de trabajo
\end{enumerate}


\section{Estructura del informe}

En este capítulo \ref{ch:intro} revisamos.... En el próximo Capítulo \ref{ch:litRev} veremos... En el Capítulo \ref{ch:metodo} ...

{\color{blue} (2 a 3 líneas que informe al lector el contenido de cada capítulo) } % Intro
% cap2.tex

\chapter{Revisión Bibliográfica}
\label{ch:litRev} % la etiqueta para referencias

{\color{blue} En esta sección de debe incluir la revisión general y especifica de la investigación
desarrollada en la tesis.
Aparte de referencias antiguas, las citas deben incluir referencias sustanciales de
los 5 últimos años que den cuenta de la situación actual del tema a investigar,
evitar el uso de páginas web o fuentes no formales de información. (Máximo 15 páginas). }

\vspace{0.5cm}

Para citar bibliografía use \citeA{pastor_sustainable_2020} o bien
\cite{pastor_sustainable_2020}, dependiendo de la redacción.

Para referirse a tablas o a figuras use Tabla \ref{tab:nombreTabla} o Figura \ref{fig:nombreFigura},
y para acrónimos como \gls{latex} definirlos en el archivo core/glosario.tex

\section{5 temas relacionados}%"Comenzar con las 5 referencias del articulo de Pastor" 
%\vspace{0.5cm}
% Agregar referencias sobre tema 1 
%(Hoeoner et al. (2018) encuentran que el compromiso ESG reduce el riesgo a la baja de las empresas, así como su exposición a un factor de riesgo a la baja.)

%Esto es el abstract traducido: Mostramos que la participación en temas ambientales, sociales y de gobernanza puede beneficiar a los accionistas al reducir los riesgos a la baja de las empresas. Encontramos que las reducciones de riesgo (medidas utilizando el valor en riesgo y los momentos parciales más bajos) varían según los tipos de participación y las tasas de éxito. La participación es más efectiva para reducir el riesgo a la baja cuando se abordan temas ambientales (principalmente el cambio climático). Además, los objetivos con grandes reducciones de riesgo a la baja muestran una disminución de los incidentes ambientales después de la participación. Estimamos que el valor en riesgo de los objetivos de participación disminuye en un 9\% de la desviación estándar después de las participaciones exitosas, en comparación con las empresas de control.

%\vspace{0.5cm}

\subsection{Compromiso de los accionistas}

Por otro lado, se ha demostrado que el compromiso en temas ambientales, sociales y de gobernanza benefecia a los inversionistas reduciendo el riesgo a la baja de las empresas, asi como tambien su exposicion a un factor de riesgo a la baja \cite{hoepner_esg_2022}. Para lograr determinar esto, se utilizaron datos exclusivos proporcionados por una gran institucion inversionista con sede en Reino Unido, siendo esta institución uno de los activistas mas influyentes en temas relacionados al ASG (ver bien a cual institucion especificamente se refiere). De acuerdo a los resultados entregados, los inversionistas principalmente se centran en temas de gobernanza, siendo esta un 51\% de la muestra, enfocandose en temas como la regulacion ejecutiva y la estructura del consejo. En segundo lugar, estan los factores ambientales siendo el tema mas referenciado sobre el riesgo climatico. Por ultimo, se encuentra el factor social, siendo la salud y seguridad, las cadenas de suministro y actos legales los principales temas a mencionar.

\vspace{0.5cm}

Debido a que las interacciones ambientales y sociales tienen una menor probabilidad de ocurrencia, es esperable que al incorporarlas en las carteras de inversiones estas reduzcan el riesgo a la baja (a que se refiere con el riesgo a la baja? podria explicarlo mejor). No obstante, esto no es tan evidente con los temas de gobernanza. Esto se debe a que las interacciones sobre temas de gobernanza podrían tener como objetivo aumentar la toma de riesgos si los gestores no diversificados asumen poco riesgo en comparación con lo óptimo para los accionistas diversificados. 

\vspace{0.5cm}
%Agregar referencias sobre tema 2 
\subsection{Subrendimiento de activos verdes respecto de cafés}
%(Hong and Kacperczyk (2009) Los activos verdes no rinden lo suficiente como los 
%activos cafes.)



Hay ciertos estudios que evidencian lo contrario sobre los beneficios que otorgan los activos verdes. Según la investigación realizada en el articulo de \citeA{hong_price_2009}, estos estudiaron en detalle las acciones de empresas que cotizan en bolsa que se dedican a la producción de alcohol, tabaco, y juegos de azar, y que normalmente son conocidos como acciones inmorales, evaluando además los efectos de las normas sociales en los mercados accionarios. 
\vspace{0.5cm}
Hoy en día hay una cierta barrera social o cultural al momento de querer invertir en estas acciones que promueven vicios en las personas, que por obviedad, tienen una puntuación ASG bajísima comparada a otros activos, igualmente hay evidencia que respalda esto gracias a la adopcion ISR (Inversion socialmente responsable) que han adoptado los administradores de instituciones como fondos de pensiones con el pasar del tiempo. Eso si, las instituciones financieras, al no querer invertir en estas acciones, incurren en un costo financiero al abstenerse en incorporarlas en sus carteras de portafolios. Por un lado, esta el costo de no poder diversificar en companias "pecadoras" que cotizan en bolsa. Por otro, estas acciones suelen ser relativamente baratas (bajos indices precio-valor en libros o precio-ganancias) en comparación con otras. Además, estas acciones "pecaminosas" suelen tener menos supervisión por parte de los analistas financieros en comparaciones a otras que no están enfocadas en vicios. Es por esto mismo que estas acciones, tienden a tener rendimientos esperados mas altos que las otras acciones, que además enfrentan un mayor riesgo de litigio gracias a las normas sociales que están presentes hoy en día.

\vspace{0.5cm}

%Agregar referencias sobre tema 3 
\subsection{Activos verdes superan a los cafés, utilizando definiciones alternativas}
%(Algunos estudios encuentran el resultado opuesto, que los activos verdes superan al cafe, utilizando definiciones alternativas de verde y café)
%Podría hablar de Firms perform better if they are better governed, judging by employee satisfaction (Edmans, 2011)
En la realidad y como va avanzando el mundo con respecto a los factores ESG, se ha demostrado lo opuesto a lo anterior mencionado, evidenciando que las empresas verdes superan en rendimiento a largo plazo a los cafés en diversas definiciones y mediciones de estos factores. Tal como lo estudia \citeA{edmans_does_2011}, el cual demuestra que las empresas al tener mejor gobernanza (siendo juzgadas por la satisfacción de sus empleados) suelen rendir mejor que las que no tienen buenas calificaciones en el mismo ámbito. 

De acuerdo a la evaluación del articulo de Edmans donde se evaluaron las mejores 100 empresas para trabajar en Estados Unidos, estas obtuvieron un alfa de cuatro factores de 3,5\% entre los años 1984 y 2009, estando por encima un 2,1\% sobre otras industrias. Además, estas obtuvieron ganancias sorpresas (a que se refiere con sorpresas?) mas positivas y rendimientos de anuncios significativamente mas positivos, siendo un 1,2\% a 1,7\% mas que otras firmas. Lo anterior mencionado se debe a 3 temas fundamentales que se han ido potenciando con el paso de los años; la importancia del capital humano en las empresas, teniendo en cuenta a los empleados como activos esenciales de la organización y no como objetos reemplazables los cuales pueden crear un valor sustancial en la empresa generando nuevas ideas y formando relaciones humanas; lo incapaz que ha sido el mercado para poder entregar toda la información respecto al verdadero valor de los activos intangibles (tal vez profundizar en lo que se refiere intangible), teniendo en cuenta la falta de información de los inversores y otros factores que hacen que la empresa no este realmente bien valorada en el mercado, generando que los inversores no estén conscientes de los verdaderos beneficios de invertir moralmente bien; y el efecto que se esta teniendo al invertir socialmente responsable (SRI en ingles). Con la teoría tradicional de portafolio de Markowitz en 1959 (arreglar citas), la cual indica que poner limitaciones en las decisiones del inversor (tal como sugiere SRI) reduciría los rendimientos, pues una optimización restringida no es mejor que una que no este restringida. Edmans logro relacionar que ciertos filtros  
%Moskowitz en 1972 (arreglar cita), Luck y Pilotte en 1973 y Derwall Guenster, Bauer, y Koedijk en 2005 
de SRI mejoran los rendimientos si se tratan de periodos cortos, siendo la satisfacción del empleado uno de los factores mas colacionados con la rentabilidad de los retornos, dándole así mas valor a las empresas, y por consiguiente superando a empresas cafés.


\vspace{0.5cm}
%Agregar referencias sobre tema 4 
\subsection{Las inversiones socialmente responsables de una empresa aumentan la lealtad de sus clientes}
%(Albuquerque et al. (2019) construyen un modelo en el cual las inversiones socialmente responsables de una empresa aumenten la lealtad de los clientes.)
Lo anterior también es consistente con lo investigado por \citeA{albuquerque_corporate_2019} los cuales lograron construir un modelo en donde se logra reflejar que las inversiones socialmente responsables (ISR), en empresas que invierten en tener una mayor diferenciación en sus productos, permite un aumento en la lealtad de los clientes, dándole así mas poder a las empresas de poder fijar precios, y de esta manera las empresas tienen estadísticamente y económicamente menos riesgo sistemático y mas valor de mercado, para empresas con un alto nivel de ISR. Esto es representado por un análisis con la Q de Tobin, donde se analiza el el efecto de ISR en el valor de la empresa, dando como resultado una relación positiva entre la Q de Tobin y la ISR, siendo mas fuerte con empresas con mayor diferenciación de productos.
\vspace{0.5cm}
%Agregar referencias sobre tema 5 
\subsection{Las inversiones en ASG tienen un impacto social positivo}
%(Oehmke and Opp (2020)La inversión en ESG tiene un impacto social positivo, siento que el paper es un poco alejado a lo de esta investigacion pero igual lo mencione)

De manera significativa, los activos que están bajo gestión en fondos socialmente responsables han ido creciendo con el tiempo, gracias a que los inversores buscan aumentar su asignación en activos con puntuaciones altas de ASG. Esto es por medio de cambios en los gustos que se están realizando en los clientes e inversores, dando como resultado que los activos verdes estén superando a los activos marrones a largo plazo. En consecuencia a esto, el tamaño de la industria en invertir en estos activos ASG, así también como los alfas de los inversores, dependen directamente de que tanto priorizan a las empresas que sigan estas políticas, de tal forma de sacrificar un poco de rendimiento, obteniendo un excedente de inversión al darse el gusto de elegirlas, generando un impacto social positivo. Esto es consistente con lo mencionado por \citeA{oehmke_theory_2023}, se realiza un estudio caracterizando las condiciones donde un fondo de inversión socialmente responsable conduce a distintas empresas a reducir externalidades, por medio de restricciones de financiamiento y coordinación entre agentes, generando un impacto y producción mas limpia en la inversión. Esto se aleja un poco a lo nuestro, pero enfatiza el impacto social que se genera, por medio de preferencias en tenencias mas verdes.

\vspace{0.5cm}

%Agregar referencias sobre tema 6 
\subsection{La frontera eficiente de ASG}
%(Pedersen (2021) “Responsible investing: The ESG-efficient frontier”)


Con el avance que ha tenido la inversión teniendo en cuenta los factores ASG, han ido existiendo distintas formas de poder valuar los verdaderos beneficios y costos que podría traer, influyendo de esta forma, en la elección de la cartera de activos y sus precios en equilibrio. Es por esto que \citeA{pedersen_responsible_2021}, logro desarrollar una teoría en la cual las puntuaciones ASG de cada activo se dividía en dos roles. Primero en entregar información sobre los principios de la compañía, y segundo, ayudar en las preferencias del inversor. La solución propuesta por Pedersen es una frontera eficiente ASG, que indica el ratio de sharpe mas lógrale según la cartera determinada. Para poder lograr esto se modelaron tres tipos de inversores:  Primero están los que si tienen conciencia ESG (Tipo-A), los cuales maximizan su utilidad, pero tienen un grado de atención a los factores de ASG al momento de tomar decisiones sobre el riesgo y rendimiento utilizado, eligiendo así su cartera con el mayor ratio de sharpe posible. Luego, están los motivados por ASG (Tipo-M), los que utilizan información exclusivamente ASG para formar sus portafolios y buscan con esto formar una cartera a la derecha de cartera tangente donde se encuentra la frontera eficiente ASG, buscando un equilibrio óptimo entre un gran rendimiento del portafolio, bajo riesgo y altos puntaciones de los factores, explicándose en una optimización a las puntaciones elevadas y el ratio de Sharpe. Por ultimo, están los que no tienen conciencia a los factores ASG (Tipo-U) y solo buscan maximizar su utilidad de media-varianza condicional, por lo que pueden elegir una cartera por debajo de la frontera eficiente ASG, pues se basan en menos información sin contar estos criterios.

\vspace{0.5cm}

%Tendre que hablar de la regulacion estado unidense que analizaremos aca en la introduccion tambien?
\subsection{Regulación Financiera/ASG}

De acuerdo con los reguladores que existen hoy en día, podemos enfocarnos tanto en Europa como en Estados Unidos, siendo la primera la mas avanzada y la pionera en emitir nuevas normas basados en los factores ASG. La Directiva de Reporte de Sostenibilidad Corporativa (DRSC o CSRD por sus siglas en ingles) es la ley encargada de la Unión Europea, que tiene como objetivo establecer requisitos, cada vez mas detallados y estrictos, en la producción de informes de sostenibilidad que deben emitir las empresas anexadas a esta ley, con el fin de ayudar a los inversores a tomar decisiones mas informadas y sostenibles. Sobre los beneficios que trae tener un ente regulador como la CSRD, estas son varias, tales como, atraer a inversores comprometidos con la sosteniblidad, una mayor transparencia e identificación en proyectos alineados con los factores ASG, reducción de costos y aumento de eficacia para las firmas vinculadas, fortalecimiento de reputación y de su imagen corporativa, y por ultimo, poder reconocer e identificar riesgos relacionados a los factores ASG y así reducir los costos asociados a estos riesgos. Por el lado de Estados Unidos, la entidad encargada en la regulación de estas normas es la Comisión de Bolsa y Valores (CBV o SEC por sus siglas en inglés). Esta busca, además de proteger a los inversionistas y regular la integridad del mercado de valores, fomentar las inversiones ASG entre los diferentes agentes del mercado, instándolos a evaluar la precisión y coherencia entre sus publicaciones y divulgaciones con respecto a las practicas internas de la empresa. Adicionalmente, las compañías deben garantizar que sus enfoques en inversión ASG se implementen de manera adecuada, integrando las nuevas políticas y procedimientos de manera transparente. 

\vspace{0.5cm}

De acuerdo a las diferencias \footnote{Por medio del foro Harvard Law School Forum on Corporate Governance} que existen en los reportes de la CSRD y La SEC, esta ultima tiene un enfoque mas fragmentado, centrando sus estandares en temas ASG mas especificos y de una perspectiva solo del inversor, los cuales no cubren la amplia gamma de requisitos que si cubre la CSRD  y la perpectiva tanto del inversor como la de la compañia misma. En temas medioambientales, la CSRD tiene reportes obligatorios centrados en la mitigacion del cambio climatico, asignacion de recursos y politicas que afecten la contaminacion del aire, agua, el suelo, los organismos vivos y los recursos alimentarios, en como la compañia afecta los recursos hidricos y marinos, como enfocan abordan la perdida de recursos no renovables y la regeneración de recursos renovables (haciendo enfasis en la economia circular), y por ultimo, en como la empresa afecta la biodiversidad y los ecosistemas. Con respecto a la SEC, esta solo se enfoca en el cambio climatico, haciendo enfasis en informes de emisiones de gases de efecto invernadero. A futuro se espera que ademas divulguen riesgos climaricos, impactos de estrategias y gobernanza climatica y gestion de riesgos, con tal de ampliar sus informes. Siguiendo con el factor social, la CSRD emite reportes enfocandose en la fuerza laboral abordando temas como las condiciones de trabajo, el acceso a igualdad de oportunidades y otros derechos relacionados al trabajo, tambien como las operaciones de la empresa afecta  a los trabajadores y las comunidades locales en la cadena de valor y como manejan estos riesgos, y por ultimo, como sus productos y servicios impactan en los consumidores y usuarios finales. La SEC en este estandar, solo incorpora en sus informes una vision muy general sobre como utilizan sus recursos y estrategias. Se espera que la SEC proponga una divulgación mas sustancial sobre este tema abordando el capital humano y la diversidad de empleados mas a profundidad. Por ultimo, en el factor de gobernanza, la CSRD incorpora de manera detallada en sus informes las politicas de diversidad, remuneracion y gestion de riesgos. Tambien detallan la composicion del directorio, sus reuniones y la tasa de asistencia. Adicionalmente, informan la estrategia, los alcances y procesos de la compañia, entregando informacion sobre la conducta de la compañia. La SEC ademas de entregar informacion sobre la estructura y composicion del consejo, estas entregan informacion sobre la organizacion de la supervision del consejo y la experencia de los directores en asusntos relacionados con el clima y la ciberseguridad. A estos requisitos, se les suma la divulgacion de los codigos de etica de la empresa.

\vspace{0.5cm}

Últimamente la Unión Europea \footnote{Por medio del foro Harvard Law School Forum on Corporate Governance}, por medio de la CSRD, tiene planeado emitir nuevas normas en los informes ASG, esperando que estas nuevas regulaciones provoquen un crecimiento exponencial en la cantidad de datos ASG que existen hoy en dia. Estas serán aplicadas tanto a empresas que pertenecen a la UE, como también a las que no pertenecen, por lo que varias empresas estadounidenses, que tengan operaciones en Europa, estarán obligadas a elaborar informes ASG de acuerdo a las nuevas regulaciones. Estas normas exigirán informes de mas alto nivel, abarcando a empresas tanto publicas como privadas, que previamente no estaban obligadas a realizar estos informes no financieros. Para las empresas estadounidenses, las nuevas normas obligaran a informar de manera mas amplia los temas ASG, que los requeridos según las normas actuales y propuestas por la SEC, teniendo enfasis en la divulgacion y objetivos futuros relacionados con el clima, mas en particular en sus emisiones de gases del efecto invernadero y como la junta directiva supervisan los riesgos relacionados al clima. Ademas, el consejo internacional de normas de sostenibilidad (CINS o ISSB por sus siglas en ingles) emitio dos borrades de exposicion: uno de Requisitos Generales para la divulgacion de informacion financiera relacionada con la sosteniblidad y el otro para el clima. Ambos se esperan que esten finalizados para inicios del 2023, con tal de satisfacer las necesidades de los inversores que solicitan este tipo de informacion de forma confiable y consistente. Adicionalmente, el Grupo Consultivo Europeo de Informacion Financiera (GCEIF o EFRAG por sus siglas en ingles), asociación que cumple con el objetivo de ayudar a la Comision Europea en la adopcion de normas internacionales de caracter financiero, profundizan el alcance de lo anterior mencionado, centrandose en plazos mas cortos para poder entregar los informes y  el impacto que sufriran las empresas con las nuevas regulaciones.  Cabe mencionar que las empresas que no estén cubiertas por las nuevas regulaciones, también sentirán el impacto de los nuevos requisitos si son parte de la cadena de valor de la entidad que es afectada. Por medio del avance de estas ultimas regulaciones, se incremento la adopcion de informes voluntarios \footnote{Por medio de un comunicado de EY: The evolution of the ESG reporting
landscape} ASG en un 96\% para las empresas del S\&P500 y un 81\% de las organizacion del Russell 1000, evidenciando el claro enfoque e importancia que se le estan dando a estos estandares, tanto por parte de las emrpesas como de los inversionistas.

\vspace{0.5cm}

De acuerdo a las normas anteriormente propuestas por la SEC \footnote{https://www.esgreportinghub.org/article/sec-proposed-rule}, estas les solicitaran a las empresas publicas que incorporen divulgaciones relacionadas con el clima en sus informes. Esto con el fin de que proporcionen a los inversores la transparencia necesaria para incorporar una vision mas ASG a sus inversiones. Estas divulgaciones se basan en marcos de divulgacion que muchas empresas privadas ya utilizan, como el Grupo de Trabajo sobre Divulgaciones Financieras Relacionadas con el Clima (o TCFC por sus siglas en ingles) y el Protocolo de Gases de Efecto Invernadero. Este marco de trabajo propuesto por la SEC se aplicaria tanto para empresas nacionales como a los emisores privados extranjeros, requiriendo: Supervision y gobernanza de los riesgos relacionados con el clima por parte de la junta directiva y la direccion del registrante, como los riesgos relacionados al clima es probable que tengan un impacto, tanto a corto como largo plazo, en el material a la empresa y a sus estados financieros, como estos riesgos podrian afectar a la estrategia y modelo de negocio de la empresa, como la empresa identifica, evalua y gestiona estos riesgos con tal de mitigarlos, una clara descripcion de las emisiones directas e indirectas de gases de efecto invernadero y de actividades aguas arriba y abajo en la cadena de valor del cliente en los informes, y futuras metas y objetivos relacionados con el clima que tengan en cuenta la empresa. De esta forma se espera que estas nuevas regulaciones generen una profunda reflexión y análisis a nivel mundial sobre cómo se gestionan y planifican actualmente los asuntos relacionados con el clima. Además, estos cambios propuestos continuarán moldeando las expectativas y requerimientos de los inversionistas, en particular de aquellos que tienen un enfoque en la sostenibilidad.

\vspace{0.5cm}

La Encuesta Institucional y de Informes Corporativos Globales de EY 2022 realizada por Ernst \& Young (EY) entrega un horizonte a que los reguladores, tanto la CSRD y la SEC, deben tomar direccion en los futuros informes que entregaran. Temas que tanto los inversores como los lideres financieros estan de acuerdo que falta en la efectividad en los informes corportativos: 1) La falta de evidencia o garantia de apoyo para generar confianza en la informacion ASG. 2) La desconexion entre los informes ASG y la informacion financiera convencional. 3) LA falta de informacion sobre ocmo la emrpesa crea valor a plazo junto a los estandares ASG. Es por ende que a futuro, es de vital inportancia que exista coherencia en lo anterior mencionado, para que de tal forma las empresas logren beneficiarse de que información financiera y no financiera es mas relevante para ellas y de esta forma logren crear valor a largo plazo. De esta forma también se combatirá el "lavado verde", donde se exageran o incluso son inexistentes los posibles beneficios sociales y ambientales de la estrategia de inverison ASG, de las empresas.

\vspace{0.5cm}

Llevándolo al ámbito local, Chile no se ha quedado atrás de las políticas ASG convirtiéndola en uno de los países mas importantes de latinoamérica en incorporar los principios para la inversion responsable (PIR o PRI por sus siglas en ingles)\footnote{Por medio de PRI Blog: Chile aumenta el interés en la inversión responsable con la COP25 en el horizonte}, (organización internacional apoyada por las Naciones Unidas que promueven un mercado financiero sostenible). La consolidación de la presencia del PRI en Chile se explica por el fortalecimiento de su vinculo con organizaciones fundamentales en el entorno de las inversiones del país, tales como la Bolsa de Santiago, FIAP, UNEP FI Chile, Universidades y entes reguladores. Además, PRI a guiado en el proceso de profundizar las practicas de inversión responsable a 6 compañías principales: AFP Cuprum, Banchile Inversiones, BICE Inversiones, Governart, Larrainvial y Moneda AM, con tal de promover a un mas en profundidad los criterios ASG en Chile. A partir de esto, se ha notado un rápido compromiso tanto por parte del gobierno como de las empresas en Chile para tomar un papel destacado en la región frente a los desafíos de la reducción de emisiones de carbono en la economía y los riesgos del cambio climático. Un gran ejemplo de esto es la emisión del primer Bono Verde \footnote{Por medio de ComunicarSe: Chile lanza con éxito su primer bono verde soberano} Soberano en Latinoamerica en el año 2019, obteniendo la tasa de adjudicación más reducida en comparación con otras economías emergentes en el transcurso de este año. 

\vspace{0.5cm}

Gracias a los avances y a la importancia que esta tomando en Chile los criterios ASG, la Comision para el Mercado Financiero (CMF)\footnote{Por medio de CMF: normativa que incorpora exigencias de información sobre sostenibilidad y gobierno corporativo en las Memorias Anuales} ha publicado una nueva normativa, la cual comenzara a regir en COP 26, que incorpora exigencias de información sobre sostenibilidad y gobierno corporativo en las futuras memorias anuales. La regulación sera aplicable a una variedad de entidades, incluyendo bancos, compañías de seguros, emisores de valores de oferta publica, administradoras generales de fondos y bolsas de valores. El objetivo es que las entidades reporten politicas, practicas y metas que sigan los estandares ASG, con tal de que los inversores y tambien el publico en general puedan analizar y seleccionar las inversiones que mejor protejan sus intereses y puedan identificar empresas mas capaces de detectar, medir y gestionar eficazmente sus riesgos. De esta forma la nueva Memoria Anual se estructurara en base a un enfoque de reporte integrado, que incluye requerimientos de información sobre el perfil de la entidad, su gobierno corporativo, sistema de gestión de riesgos, estrategia y modelo de negocios.


\subsection{Inatención Racional}

Con respecto a la Inatención racional, hay varias visiones respecto a lo que realmente sucede con los rendimientos de las carteras. Por un lado, esta la visión de que la falta de atención racional tiene puede disminuir la demanda de los inversores en activos riesgosos como acciones, incluso cuando se presentan pequeñas desviaciones de la plena racionalidad. Esta tendencia se explica porque los inversores que actúan con falta de atención racional asumen un mayor riesgo a largo plazo al no procesar toda la información posible, lo que los lleva a buscar rendimientos mas altos como compensanción \cite{luo_long-run_2016}. Además, \cite{miao_asset_2023}, añade que a menor demanda de activos riesgosos podría ocasionar que el mercado sea menos eficiente. Por otro lado, hay autores que indican que la falta de atención racional no necesariamente reduce la inversión sustentable. Como demuestra \citeA{pastor_sustainable_2021}, en equilibrio, los inversores disfrutan de mantener activos verdes y lo ven como cobertura ante el riesgo climático. Aun si los activos verdes ofrecen rendimientos mas bajos, estos tienen un mayor desempeño cuando los shocks positivos aumentan la demanda de ellos. Adicionalmente Pastor añade que la demanda de inversion ASG crece mas cuando los inversores tienen preferencias mas diversas para los activos ASG. Luo 2008. Tambien, se ha modelado que la forma en que la informacion se propaga en los mercados depende de la atencion, y la falta de atención a pequeños cambios puede explicar los efectos momentum en el mercado y los cambios en los precios de los diferentes activos \citeA{heinke_rational_2016}. Un ejemplo de esto, es la burbuja inmobiliaria en Estados unidos, la cual causo la gran crisis del 2008, dando a entender que los servicios financieros proporcionan capacidad de procesamiento de información. 

\vspace{0.5cm}

Se sabe que la sobre abundancia de información resulta en una "limitación de la atención", lo que plantea la pregunta clave para un inversor: ¿Como se puede gestionar eficazmente esta avalancha de datos?.
El concepto de "Inatención Racional" es una formalización que cada vez es mas popular en el mundo de las finanzas. Esta sigue la idea de que la limitada capacidad para procesar información (o "atención") podría explicar la simplicidad de las acciones humanas en comparación con las de los agentes en los modelos  económicos \cite{sims_implications_2003}. Un buen ejemplo de esto, presente en el articulo de Sims, es que los precios solo reaccionan lentamente a los choques monetarios porque las empresas destinan la mayor parte de su atención a los choques idiosincrásicos altamente volátiles. Además este modelo, plantea la noción de que la capacidad de las personas para transformar información externa en acciones se ve restringida por una limitación finita de procesamiento de datos según la teoría de Shannon (Es un canal de comunicaciones, es la velocidad máxima teórica de transferencia de datos a través de un canal, dada una cierta cantidad de ruido). De esta forma, un enfoque de inatención racional implica un comportamiento generalizado de inercia y erraticidad, la cual sugiere conexiones entre variables en el grado y la naturaleza de la inercia \citeA{Sims2010Rational}. De este mismo estudio, varios investigadores y autores proponen una visión particular de la inatención racional en distintos ámbitos de la economía.

\vspace{0.5cm}

En el sentido de como viaja la información dentro de los mercados, \citeA{heinke_rational_2016} proporciona un modelo donde se brinda evidencia empírica de que la noción de eficiencia de la información impulsada por la atención se ajusta mejor a los datos del mercado en comparación con el concepto predominante de mercados eficientes. Este modelo logra explicar cambios en los precios de los activos debido a una falta de atención hacia cambios permanente en los fundamentos económicos, mediante un componente de cambio agregado a la ecuación estudiada, llegando a un modelo resultante el cual incorpora la heterogeneidad de los agentes por medio dos maneras, por la restricción de capacidad de información de cada uno de los agentes y por la señal misma del cambio de fundamentos. Los principales resultados obtenidos en su articulo se basan en dos conceptos claves. El primer concepto se basa en el concepto de inatención racional, con tal de modelar la adquisición de información y la asignación de atención. Este concepto logro explicar el hallazgo empírico de que los cambios en los fundamentos a largo plazo requieren tiempo, antes de que se reflejen completamente en el precio. Esto también explica las estrategias de negociación, basadas en el momentum, como un resultado de acciones racionales en relación al precio, mostrando que los precios de los activos si reflejan completamente los cambios en los fundamentos solo a largo plazo. En el segundo concepto, se utilizo un marco de generaciones superpuestas que les permitió modelar procesos de agregación de información dentro de un mercado competitivo de activos. De esta forma se logro permitir heterogeneidad en las restricciones de procesamiento de información, con tal de poder analizar como los inversionistas convierten su capacidad para procesar enormes cantidades de datos en una ventaja informativa que genere grandes renidmientos.

\vspace{0.5cm}

Otra visión es la que se estudia en el articulo de \citeA{huang_rational_2007}, los cuales plantean que el costo asociado a la obtencion de  informacion lleva a que los inversores presten atencion de manera selectiva a las noticias, optando por recibirlas con una frecuencia limitada. Esta atencion selectiva, mencionada anteriormente como "Inatencion racional", hace que la estrategia de negociacion sea "miopica" en relacion con la frecuencia y precision de las noticias. Dicho en otras palabras, siempre y cuando la distribucion condicional actual sea la misma, las estrategias de negociacion optimas seran las mismas, independientemente de si se espera tener mas informacion a futuro. De esta manera, el modelo predice que el volumen de negociacion aumenta en los momentos donde hay nuevas noticias debido a que disminuye la incertidumbnre sobre la variable predictiva. Llegando a la conclusion de que la frecuencia optima de las noticias aumenta con la volatibilidad incondicional de la variable predictiva y disminuye con el costo de la informacion y la aversion al riesgo. En otras palabras, cuando un inversionista es averso al riesgo o tiene un horiznote de inverison mayor, este elige actualizacion periodicas de noticias menos frecuentes pero mas precisas, asumiendo el costo que se puede llegar a asumir.

\vspace{0.5cm}

%parrafo conclusivo%

En este contexto, se persigue mantener la estructura propuesta por el enfoque de \cite{pastor_sustainable_2021}, el cual se orienta hacia la modelación de las inversiones en el mercado, considerando a las empresas que adoptan políticas ASG. El objetivo principal es demostrar que las carteras de inversión que incorporan activos vinculados a estas prácticas ```verdes" superan en rendimiento a aquellas que incluyen activos relacionados con prácticas menos sostenibles, lo que también refleja un cambio en las preferencias de los consumidores hacia acciones más sostenibles.

Además, se busca integrar el concepto de inatención de los individuos al momento de asimilar una nueva regulación. \cite{heinke_rational_2016} han desarrollado un modelo que explica las variaciones en los rendimientos de los activos en respuesta a cambios en los fundamentos económicos, en este caso, la regulación. Esto se explica en función de la cantidad de información que un agente puede captar y la señal proporcionada por la regulación en sí.

De esta forma, el trabajo de \cite{pedersen_responsible_2021} representa un esfuerzo por combinar los aspectos de inatención y sostenibilidad en un solo marco. Propone la construcción de una frontera eficiente ASG que indica el ratio de Sharpe más alcanzable según un portafolio específico, definido de manera exógena por diferentes tipos de inversores.

En este sentido, el propósito fundamental de la presente investigación es ampliar el modelo desarrollado por \cite{pedersen_responsible_2021} en un entorno en el que los agentes determinan de manera endógena su grado de inatención. Esta ampliación permitirá la elaboración de una evaluación que examinará cómo han evolucionado las preferencias por la sostenibilidad cuando se han producido modificaciones en la regulación referente a la forma en que se comunica la información. % Rev. biblio
% cap3.tex

\chapter{Metodología}
\label{ch:metodo} % la etiqueta para referencias

{\color{blue} En esta sección de debe incluir el diseño experimental y metodología general del trabajo que fueron utilizados, su explicación, así como el detalle de los materiales, equipos y metodologías teóricas y experimentales utilizadas. (Máximo 15 páginas). }

El objetivo de este estudio es desarrollar un nuevo enfoque que incluya las funciones de utilidad de los inversores, las cuales están condicionadas por un rendimiento financiero considerando factores ASG.  Siguiendo la línea de investigación de \cite{pastor_sustainable_2021}, se busca incorporar los hallazgos previos de \cite{heinke_rational_2016}, así como los de \cite{pedersen_responsible_2021}, en relación con la adquisición de información acerca de los rendimientos ASG, tomando en consideración la noción de inatención racional.

\section{Un modelo de varios activos y retornos ESG}

Sigamos el modelo de \citeA{pastor_sustainable_2021}, donde se tienen $N$ firmas con un retorno sobre el retorno del activo libre de riesgo $r_f$ denotado por $\tilde r_n$ para cada $n=1,...,N$. El vector de retornos de todo el modelo se puede escribir de la siguiente forma

$$\tilde r=(\tilde r_1,...,\tilde r_N)=(\mu_1,...,\mu_N)+(\tilde\epsilon_1,...,\tilde\epsilon_N)=\mu+\tilde\epsilon.$$

Esto quiere decir que la firma $n$ tiene un retorno mañana igual a $\tilde r_n=\mu_n+\tilde\epsilon_n$. En otras palabras, en promedio la firma $n$ tiene un retorno $\mu_n$ pero dado que se realizó un estado de la naturaleza que induce un error $\tilde\epsilon_n$, entonces el retorno es la suma de ambos.

Pregunta: Si cada valor de $\tilde\epsilon_n$ posible refleja un estado de la naturaleza y existe una cantidad $S>0$ de estados posibles: ¿Cómo se puede escribir el vector de retornos de los activos $R_n=(R_n(1),...,R_n(S))$?\\

El vector de retornos de los activos $R_n=(R_n(1),...,R_n(S))$ se escribiría de acuerdo a la siguiente ecuación (\ref{eq:Rn}):

\begin{eqnarray}
R_n&=&(\mu_1 + \tilde\epsilon_1,...,\mu_N + \tilde\epsilon_N)\label{eq:Rn}
%\\
%&\text{otra línea}&\nonumber
\end{eqnarray}

Además de los beneficios pecuniarios, las empresas también entregan un beneficio al impacto social. Cada empresa $n$, tiene una característica ASG, la cual es capturada por $g_n$, siendo para empresas ``verdes'' positiva si tienen externalidades positivas y para empresas ``cafés'' negativa si tienen externalidades negativas.\\

También existe un continuo de agentes los cuales comercian con las acciones de las empresas y el activo libre de riesgo. Denotemos por $X_i$ a un vector de $N x 1$ cuyo elemento n-ésimo es la fracción de la riqueza del agente $i$ invertida en el activo $n$. La riqueza del agente $i$ en el segundo periodo ($t = 1$) es:

$$W_{1i}=W_{0i}(1 + r_f + X_i' \tilde r)$$

donde $W_{0i}$ representa la riqueza inicial del agente i (es decir, en $t=0$). Además de su riqueza, los agentes obtienen una utilidad (valoración subjetiva de la riqueza) dependiendo si utilizan acciones ``verdes'' o ``cafés''. Cada agente i tiene una utilidad exponencial:

$$V(\tilde W_{1i}, X_i)=-e^{-A_i\tilde W_{1i}-b_i'X_i}=-\exp\left(-A_i\tilde W_{1i}-b_i'X_i\right)$$

Donde $A_i$ es la aversión al riesgo que tiene el agente $i$ y $b_i$ es un vector $N \cdot 1$ de beneficios otorgados por ASG que el agente obtiene gracias a la tenencia de esas acciones. El vector de beneficios tiene componentes específicos para cada agente y empresa:

$$b_i=d_ig$$

Donde $g$ es un vector en $\mathbb{R}^N$ cuyo elemento n-ésimo es $g_n$ y $d_i \geq 0$ es un escalar el cual mide el grado de ``gusto'' que tiene el agente $i$ por los activos ASG. 

En este modelo, los agentes al momento de elegir sus carteras óptimas en $t=0$, consideran los precios de los activos, y la distribución de sus rendimientos como dados. Con tal de obtener la derivada de primer orden para $X_i$, y de esta manera conocer los pesos del portafolio del agente i, se calcula la esperanza de la utilidad (ecuación de mas arriba) Además, recordando que $\tilde r$ distribuye normal, $\tilde r \sim N(\mu, \Sigma)$ Se obtiene que:

\begin{eqnarray}
 E\left\lbrace V(\tilde W_{1i}, X_i) \right\rbrace &=& E\left\lbrace-e^{-A_i\tilde W_{1i}-b_i'X_i} \right\rbrace\nonumber\\
 &=& E\left\lbrace-e^{-A_i\left[W_{0i}(1 + r_f + X_i' \tilde r)\right]-b_i'X_i} \right\rbrace\nonumber\\
 &=& -e^{-a_i(1+r_f)}E\left\lbrace e^{-a_iX_i'\left[\tilde r + \frac{1}{a_i}b_i)\right]}\right\rbrace\nonumber\\
 &=& -e^{-a_i(1+r_f)} e^{-a_iX_i'\left[E(\tilde r)+\frac{1}{a_i}b_i\right]+\frac{1}{2}a_i^{2}X_i'Var(\tilde r)X_i}\nonumber\\
 &=& -e^{-a_i(1+r_f)} e^{-a_iX_i'\left[\mu+\frac{1}{a_i}b_i\right]+\frac{1}{2}a_i^{2}X_i'\Sigma X_i}
\end{eqnarray}

Donde $a_i \equiv A_iW_{01}$ es la aversión al riesgo relativo del agente $i$. Teniendo $\mu$ y $\Sigma$ como dados. Se deriva con respecto a $X_i$, obteniendo la condición de primer orden:

\begin{eqnarray}
-a_i\left[\mu + \frac{1}{a_i}b_i\right]+\frac{1}{2}a_i^{2}(2\Sigma X_i) = 0
\end{eqnarray}
Donde se obtienen los pesos de la cartera del agente $i$
\begin{eqnarray}
X_i=\frac{1}{a_i} \Sigma^{-1}\left(\mu + \frac{1}{a_i} b_i\right)
\end{eqnarray}

Siendo, $a_i = A_iW_{0i}$ la aversión al riesgo del agente i. Luego, se define $\omega_i$ como la proporción de la riqueza inicial del agente i con respecto a la riqueza total inicial del mercado: $\omega_i \equiv W_{0i}/W_0$, donde $W_0 + \int_i\omega_iW_{0i}di$. Debido a que \citeA{pastor_sustainable_2021}, asume una posición agregada cero en el activo sin riesgo, el equilibrio de mercado requiere que $w_m$, que es el vector $N x 1$ de ponderaciones de la cartera de mercado de activos, cumpla con las siguientes condiciones:

\begin{eqnarray}
 w_m&=&\int_i\omega_iX_idi\nonumber\\
&=& \frac{1}{a_i} \Sigma^{-1}\mu+ \frac{\bar d}{a^2} \Sigma^{-1}g  
\end{eqnarray}

Donde $\bar d \equiv \int_i\omega_i d_idi \geq 0$  es la media ponderada por riqueza de los gustos ASG $d_i$ entre todos los agentes.

\section{El factor ASG}

Con tal de poder obtener información sobre el rendimiento ex-post frente al rendimiento ex-ante de las acciones ``verdes'' frente a las ``cafés'', se introduce un factor ASG, el cual es empíricamente identificable, además de que esta estrechamente relacionado con las cartera ASG, manteniendo la fijación de precios de dos factores. Se define el factor ASG como:

$$\tilde f_g = \left(\frac{1}{g_g}\right)\tilde{r}_g$$

Donde $\tilde f_g$ es el rendimiento excedente de una cartera con componente ASG, según el signo y el valor de $g_g$.

Definiendo el vector de betas con respecto a $\tilde r_g$ como:

$$\beta_g = 1/g_g g$$

Y los rendimientos excedentes de dos factores según la regresión:

$$\tilde{r} = \beta_m\tilde{r}_m + \beta_g\tilde{r}_g + \tilde{v}$$

Podemos reescribir el modelo de la siguiente manera:
$$\tilde{r} = \beta_m\tilde{r}_m + g \tilde{f}_g + \tilde{v}$$

De tal forma que las ponderaciones de los activos en el factor ASG y sus betas ASG, son simplemente sus características ASG, $g$. Por lo que un cumplimiento mas alto de los esperado de $\tilde f_g$ impulsa los rendimientos de las acciones ``verdes'' y deprime los ``cafés''.

\section{Cambios en Precios de Activos por Falta de Atención en la Regulación}

Con tal de poder explicar y modelar la inatención que tienen los inversores con una noticia ASG, primero se explicara como el flujo de información de aquella señal llega al receptor. \cite{heinke_rational_2016} modelan una economía con $N$ activos y el flujo de los dividendos de cada activo como $n \epsilon N$, siguiendo un proceso estocástico con una media determinista $\mu_n$ y una varianza $\sigma^{2}_n$. Por lo tanto, el flujo de dividendos se puede describir como: 

$$d_{n,t} = \mu_n + \sigma_n \epsilon_{n,t}$$

Donde $\epsilon_{n,t} \sim \mathcal{N}(0, 1)$ representa la variable aleatoria la cual distribuye normal, con media 0 y varianza 1. Acá se presenta el concepto de generaciones traslapadas, donde en cada periodo $t$, la generación anterior $t-1$ ya se encuentra en la economía poseyendo los activos y una generación $t$ nace, compuesta por una serie de agentes idénticos distribuidos de manera uniforme en el intervalo unitario con una masa poblacional constante de 1. De esta forma, los agente de la generación $t$ pueden comprar el activo y ganar el dividendo, que se destinara a su consumo después de pagar por los activos. Para el periodo $t+1$, la generación $t$ vende el activo y consume el efectivo recibido. Por lo que, poseer un activo significaría ahorrar para el consumo en el siguiente periodo. Con tal de poder diseñar como llega esa noticia al agente, se modela la estructura de señal de un activo $n$ en el periodo $t$ que el agente $i$ elige observar, tomando en conjunto todas las posibles estructuras de señal $\Gamma$ y consta de los futuros dividendos obtenibles, mas un ruido $ \tilde\sigma^{i}_n \psi^{i}_{nt}$, donde $\tilde\sigma^{i}_n$ en el componente que mide el nivel de ruido de la noticia y $\psi^{i}_{nt}$ sigue una distribución normal con media 0 y varianza 1. Con esto, se puede modelar la precisión de la señal, la cual es una función de la cantidad de información contenida en la señal, medida por la información mutua promedio que la señal contiene sobre el dividendo. Debido a que el agente $i$ no tiene una capacidad infinita de procesamiento de información, existe un limite superior $k^{i}$ en la cantidad de información procesada $I(.)$. De esta forma se modela la ecuación de procesamiento de la información de acuerdo a una señal de la siguiente forma:

$$S^{i}_{n,t} = \mu_n + \sigma_n\epsilon_{nt} + \tilde\sigma^{i}_n \psi^{i}_{nt}$$

Donde $S^{i}_{n,t}$ es el vector de todas las señales elegidas por el agente $i$.

Además, se añade la siguiente restricción de capacidad de información procesada por el individuo $i$:

$$I (d_t; s^i_t) \leq \kappa^i $$

Donde $d_t$ es el vector de los procesos estocásticos de todos los dividendos
También se especifica la tasa de sustitución intertemporal del agente dada por $\beta$.

De esta forma, se puede modelar el problema de decisión de dos estados del agente de la siguiente forma.

$$\max_{s_t^i \in \Gamma} E [u(c_t^i; s_t^i) + \beta u(c_{t+1}^i; s_t^i) | s_t^i]$$

Sujetos a las siguientes restricciones:

$$I (d_t; s^i_t) \leq \kappa^i $$
$$ q^{i*}_{t+1} = \arg \max_{q_{t+1}^i} E [u(c_t^i; s_t^i) + \beta u(c_{t+1}^i; s_t^i) | s_t^i]$$
$$C^{i}_t = q^{i*}_{t+1} (d_t - p_t)$$
$$C^{i}_{t+1} = q^{i*}_{t+1} p_{t+1}$$

En el primer estado, se modela la decisión del agente sobre la estructura de señal que quiere recibir (4), en base a su capacidad de poder recibir información (5). En el segundo estado, el decide su estrategia de negociación (6) de acuerdo a la señal recibida por la estructura elegida y sus restricciones presupuestarias (7) y (8).

Con lo anterior modelado, se quiere llevar la siguiente función objetivo:

$$\max_{s_t^i \in \Gamma} E [u(c_t^i; s_t^i) | s_t^i]$$

Donde la ecuación de utilidad este diseñado según lo modelado en \cite{pastor_sustainable_2021}, donde se modela la utilidad esperada del agente $i$ según la relación $W_{1i}=W_{0i}(1 + r_f + X_i' \tilde r)$, con $r$ siguiendo una distribución normal, de tal forma que: 

$$E{V(\tilde W_{1i}, X_i)} = E { -e^{-A_i\tilde W_{1i}-b_i'X_i}} =-\exp\left(-A_i\tilde W_{1i}-b_i'X_i\right) $$

Con esto modelo se plantean dos preguntas:

\begin{enumerate}
    \item ¿La señal en el modelo de Heinke y Warmuth, entrega información sobre retorno pecuniario?
    \item ¿La función objetivo propuesta, entregaría información sobre retorno pecuniario y retorno ASG?
\end{enumerate}

Con esto, también cabe recalcar las similitudes y diferencias que estos dos modelos presentan. Ambos presentan un modelo continuo de agentes. \cite{heinke_rational_2016} presenta en su modelo un consumo intertemporal, mientras que Pastor no. ( y dos mas que no se leen el foto que saque)

\newpage
%EJEMPLO:

%Aunque $M1$ es un modelo fácil de entender, es bastante difícil de manejar con solvers tradicionales, puesto que es un problema muy no lineal. Alternativamente, se puede definir fácilmente una ecuación de Bellman para $M1$, considerando que es un proceso de decisión secuencial [Nemhauser y Wolsey, 2014]. Aún así, esto podría ser dificultoso para resolver con instancias de gran tamaño (incluso de tamaño mediano), debido a la maldición de la dimensionalidad. No obstante, se propone una transformación preliminar de $M1$ a un modelo de Programación Lineal Entero Mixto (MILP) $M2$, el cual es matemáticamente equivalente y que mantiene los mismos parámetros y se agrega un nuevo grupo de variables binarias. Sea $x_{it} = 1$ si se concede el préstamo al consumidor $i \in I$ en el periodo $t \in \{1,...,T\}$, 0 en otro caso.  \\


%$$ \begin{array}{r l l}
	%\displaystyle\max_{\vb{x}, \tau, \vb{s}} & z(\vb{x}, \tau, \vb{s}) &   									\\
	%\text{s.t.}&  s_{t-1} - s_{t} + \gamma_{t} - \displaystyle\sum_{i \in I} a_{it} x_{it} = 0	& \forall t \in \{1,...,T\}		\\
					%&  s_{0} = B & 	                                                                                         	\\
					%& x_{it} \leq \dfrac{\tau_{t}}{Pod_{it}} & \forall i \in I , \text{  } \forall t \in \{1,...,T\}                 \\
					%& x_{it} \in \{0, 1\} & \forall i \in I , \text{  }  \forall t \in \{1,...,T\}                                    \\
					%& \tau_{t} \in \left[0,1 \right] , s_{t} \geq 0 & \forall t \in \{1,...,T\}                                            \\
					
%\end{array} $$

%\newpage

%\[ z(\vb{x}, \tau, \vb{s}) = \begin{cases} z_1(\vb{x}, \tau, \vb{s}) = \sum_{t=1}^{T}\left(- \alpha\tau_t + \sum_{i \in I} a_{it} x_{it}\right) & \text{Si la IF es con fines de lucro} \\ z_2(\vb{x}, \tau, \vb{s}) = \sum_{t=1}^{T} \left(- \alpha\tau_t + \sum_{i \in I} x_{it}\right) & \text{Si la IF es sin fines de lucro} \end{cases} \] \\

%$M2$ es un MILP; por lo tanto, se espera que solvers como Gurobi, Cplex o SCIP (entre otros) puedan manejarlo para instancias grandes. En particular, se resolverá usando SCIP en combinación con Python. Para la solución de este problema, se utilizará el conjunto de datos de Bravo et al. [2013], que tiene miles de prestatarios que solicitan préstamos en un tiempo determinado. \\

%Sin embargo, $M2$ asume que la cantidad de clientes que llegan a solicitar un préstamo y la montos solicitados son conocidos, luego se propone un nuevo modelo con parámetros estocásticos.

%\section{Modelo $M3$: Modelo con el número de clientes como variable aleatoria}

%En $M1$ y $M2$ definimos la demanda como un parámetro, indicando el número de prestatarios $|I|$ y los montos solicitados $a_{it}$, pero ahora en este modelo $M3$ se definirán como esta cantidad de clientes como una variable aleatoria y los montos solicitados como un parámetro estocástico. \\

%Se asume que en un periodo $t \in \{1,...,T\}$, los escenarios $k \in \{1,...,K\}$ representan rangos de probabilidad de default en orden incremental. Por lo tanto, si $k_1, k_2 \in \{1,...,K\}$ son tales que $k_1 < k_2$, entonces $pod^{k_1}< pod^{k_2}$. El número de clientes que llegan a solicitar un préstamo es ahora $m_{t}^{k}$ y los montos solicitados por cada prestatario del rango $k$ es $a_{t}^{k}$. Notemos, una vez que un prestatario solicita un préstamo, la IF puede estimar sus probabilidades de default. Se pueden aplicar los procesos estándar de estimación de PD definidos por Baesens et al. [2017b] y Thomas [2009].

%Sea $x_{t}^{k} = 1$ si se concede préstamos a los consumidores que pertenecen al rango $k \in \{1,...,K\}$ en el periodo $t \in \{1,\cdots,T\}$, 0 en otro caso. 

%Luego notemos que $x_{t}^{k}$ y $\tau_t$ son variables de control, las variables de estado $s_t$ y $m_{t}^{t}$ es una variable aleatoria. Adicionalmente, definimos la función de beneficio $g_t(x_t,\tau_t, m_t)$ y de transición de estado $f_t(x_t, s_{t-1},m_{t})$ como:


%\[ g_t(x_t,\tau_t, m_t)  = \begin{cases} g_t^1(x_t,\tau_t,m_t) = \sum_{k=1}^{K}\left(- \frac{\alpha}{K}\tau_t +  \mathbb{E}\{a_{it}\} m_{t}^{k} x_{t}^{k}\right) & \text{Si la IF es con fines de lucro} \\ g_t^2(x_t,\tau_t, m_t) = \sum_{k=1}^{K}\left(- \frac{\alpha}{K}\tau_t +  m_{t}^{k} x_{t}^{k}\right) & \text{Si la IF es sin fines de lucro} \end{cases} \] 
 
%$$f_t(x_t,s_{t-1},m_t)= s_{t-1} + \gamma_t - \displaystyle\sum_{k=1}^{K}\mathbb{E}\{a_{it}\} m_{t}^{k} x_{t}^{k}$$

%Definiendo el problema en términos de maximización del beneficio esperado

%$$ \begin{array}{r l l}
	%\displaystyle\max_{\vb{x}, \tau} & \mathbb{E} \left\lbrace \displaystyle\sum_{t=1}^{T} g_t(x_t,\tau_t, m_t) \right\rbrace &   									\\
	%\text{s.t.}&  s_{t} = f_t(x_t,s_{t-1},m_t)	& \forall t \in \{1,...,T\}		\\
	%				&  s_{0} = B &                                                                                   	\\
					%& \tau_{t} = \displaystyle\sum_{k=1}^{K} x_{t}^{k}({pod_{t}^k - pod_{t}^{k-1}}) & \forall k \in \{1,...,K\}                 \\
					
					%& x_{t}^{k-1} \geq x_{t}^{k} & \forall t \in \{1,...,T\}  , \text{  } \forall k \in \{1,...,K\} \\
					
					%& x_{t}^k \in \{0, 1\} & \forall t \in \{1,...,T\}  , \text{  } \forall k \in \{1,...,K\} \\                                      
					%& \tau_{t} \in \left[0,1 \right] , s_{t} \geq 0 & \forall t \in \{1,...,T\}                \\
					
%\end{array} $$




 % Metodología
\chapter{Resultados y Análisis}
\label{ch:resultados} % la etiqueta para referencias

{\color{blue} En esta sección de debe incluir todos los resultados obtenidos, incluyendo figuras, gráficas y tablas. Estos resultados deben ser explicados, analizados y discutidos respecto del estado del arte de la investigación. (Máximo 20 páginas). } % Resultados
\chapter{Conclusiones}
\label{ch:conclu} % la etiqueta para referencias

{\color{blue} En esta sección de debe incluir todas las conclusiones del trabajo que den respuesta a las diferentes preguntas de investigación, hipótesis y objetivos propuestos. 
La sección de conclusiones no incluye análisis y discusiones de resultados realizados en las secciones anteriores (3 a 4 páginas). }

\section{Conclusiones teóricas}
\section{Conclusiones prácticas}
\section{Investigación futura}

 % Conclusiones

\cleardoublepage

\addcontentsline{toc}{chapter}{Nomenclatura}
\chapter*{Nomenclatura}
%\addcontentsline{toc}{chapter}{Nomenclatura}

%\printglossary[type=\acronymtype]
%\printglossary[title=Special Terms, toctitle=Nomenclatura]

GCD: greatest common divisor

% Ingenieros prefieren la bibliografia antes que los anexos

\cleardoublepage

\addcontentsline{toc}{chapter}{Referencias}
%{\color{blue} En esta sección de debe incluir todas las referencias utilizadas en la tesis. Se debe evitar el uso de páginas web o fuentes no formales de información. (Máximo 15 páginas)

%\vspace{0.5cm}

%Sobre el formato de citas, se debe usar el formato APA, esto es: ``Según datos de Fernández (1999), un 110\% de los conductores no alcanza a frenar... En otro estudio del mismo autor (Fernández, 2000) se establece... Otras investigaciones (Fernández y Vergara, 2000) concluyen además... El trabajo de Fernández et al. (2001) confirma que...''

%\vspace{0.5cm}

%Para mayor información consultar: \url{https://uandes.libguides.com/citasyreferencias} 
%}
\bibliographystyle{newapa}
\bibliography{bibliografia}

\appendix % seguido de archivos de apendices

\cleardoublepage

% Escribe 'Anexos' en el Indice General
\addcontentsline{toc}{chapter}{Anexos}


% anexo_a.tex

\chapter{Primer anexo o apéndice}
\label{app:nombreApendice}

{\color{blue} En esta sección de debe incluir todo el material suplementario utilizado en la tesis, así como los artículos publicados (no incluir enviados o aceptados). }


\section{Sección bla bla ba}

TEXTO
    % archivo de apendice

% \input{ruta_al_Archivo/(apendice_b}    % otro archivo de apendice

\label{end}
\end{document}

%%%%%%%%%%%%%%%%%%%%%%%%%%%%%%%%%%%%%%%%%%%%%%%%%%%%%%%%%%%%
