% cap2.tex

\chapter{Revisión Bibliográfica}
\label{ch:litRev} % la etiqueta para referencias

{\color{blue} En esta sección de debe incluir la revisión general y especifica de la investigación
desarrollada en la tesis.
Aparte de referencias antiguas, las citas deben incluir referencias sustanciales de
los 5 últimos años que den cuenta de la situación actual del tema a investigar,
evitar el uso de páginas web o fuentes no formales de información. (Máximo 15 páginas). }

\vspace{0.5cm}

Para citar bibliografía use \citeA{pastor_sustainable_2020} o bien
\cite{pastor_sustainable_2020}, dependiendo de la redacción.

Para referirse a tablas o a figuras use Tabla \ref{tab:nombreTabla} o Figura \ref{fig:nombreFigura},
y para acrónimos como \gls{latex} definirlos en el archivo core/glosario.tex

\section{5 artículos relacionados}%"Comenzar con las 5 referencias del articulo de Pastor" 
%\vspace{0.5cm}
% Agregar referencias sobre tema 1 
%(Hoeoner et al. (2018) encuentran que el compromiso ESG reduce el riesgo a la baja de las empresas, así como su exposición a un factor de riesgo a la baja.)

%Esto es el abstract traducido: Mostramos que la participación en temas ambientales, sociales y de gobernanza puede beneficiar a los accionistas al reducir los riesgos a la baja de las empresas. Encontramos que las reducciones de riesgo (medidas utilizando el valor en riesgo y los momentos parciales más bajos) varían según los tipos de participación y las tasas de éxito. La participación es más efectiva para reducir el riesgo a la baja cuando se abordan temas ambientales (principalmente el cambio climático). Además, los objetivos con grandes reducciones de riesgo a la baja muestran una disminución de los incidentes ambientales después de la participación. Estimamos que el valor en riesgo de los objetivos de participación disminuye en un 9\% de la desviación estándar después de las participaciones exitosas, en comparación con las empresas de control.

%\vspace{0.5cm}

\subsection{Compromiso de los accionistas}

Por otro lado, se ha demostrado que el compromiso en temas ambientales, sociales y de gobernanza benefecia a los inversionistas reduciendo el riesgo a la baja de las empresas, asi como tambien su exposicion a un factor de riesgo a la baja \cite{hoepner_esg_2022}. Para lograr determinar esto, se utilizaron datos exclusivos proporcionados por una gran institucion inversionista con sede en Reino Unido, siendo esta institución uno de los activistas mas influyentes en temas relacionados al ASG (ver bien a cual institucion especificamente se refiere). De acuerdo a los resultados entregados, los inversionistas principalmente se centran en temas de gobernanza, siendo esta un 51\% de la muestra, enfocandose en temas como la regulacion ejecutiva y la estructura del consejo. En segundo lugar, estan los factores ambientales siendo el tema mas referenciado sobre el riesgo climatico. Por ultimo, se encuentra el factor social, siendo la salud y seguridad, las cadenas de suministro y actos legales los principales temas a mencionar.

\vspace{0.5cm}

Debido a que las interacciones ambientales y sociales tienen una menor probabilidad de ocurrencia, es esperable que al incorporarlas en las carteras de inversiones estas reduzcan el riesgo a la baja (a que se refiere con el riesgo a la baja? podria explicarlo mejor). No obstante, esto no es tan evidente con los temas de gobernanza. Esto se debe a que las interacciones sobre temas de gobernanza podrían tener como objetivo aumentar la toma de riesgos si los gestores no diversificados asumen poco riesgo en comparación con lo óptimo para los accionistas diversificados. 

\vspace{0.5cm}
%Agregar referencias sobre tema 2 
\subsection{Subrendimiento de activos verdes respecto de cafés}
%(Hong and Kacperczyk (2009) Los activos verdes no rinden lo suficiente como los 
%activos cafes.)



Hay ciertos estudios que evidencian lo contrario sobre los beneficios que otorgan los activos verdes. Según la investigación realizada en el articulo de \citeA{hong_price_2009}, estos estudiaron en detalle las acciones de empresas que cotizan en bolsa que se dedican a la producción de alcohol, tabaco, y juegos de azar, y que normalmente son conocidos como acciones inmorales, evaluando además los efectos de las normas sociales en los mercados accionarios. 
\vspace{0.5cm}
Hoy en día hay una cierta barrera social o cultural al momento de querer invertir en estas acciones que promueven vicios en las personas, que por obviedad, tienen una puntuación ASG bajísima comparada a otros activos, igualmente hay evidencia que respalda esto gracias a la adopcion ISR (Inversion socialmente responsable) que han adoptado los administradores de instituciones como fondos de pensiones con el pasar del tiempo. Eso si, las instituciones financieras, al no querer invertir en estas acciones, incurren en un costo financiero al abstenerse en incorporarlas en sus carteras de portafolios. Por un lado, esta el costo de no poder diversificar en companias "pecadoras" que cotizan en bolsa. Por otro, estas acciones suelen ser relativamente baratas (bajos indices precio-valor en libros o precio-ganancias) en comparación con otras. Además, estas acciones "pecaminosas" suelen tener menos supervisión por parte de los analistas financieros en comparaciones a otras que no están enfocadas en vicios. Es por esto mismo que estas acciones, tienden a tener rendimientos esperados mas altos que las otras acciones, que además enfrentan un mayor riesgo de litigio gracias a las normas sociales que están presentes hoy en día.

\vspace{0.5cm}

%Agregar referencias sobre tema 3 
\subsection{Activos verdes superan a los cafés, utilizando definiciones alternativas}
%(Algunos estudios encuentran el resultado opuesto, que los activos verdes superan al cafe, utilizando definiciones alternativas de verde y café)
%Podría hablar de Firms perform better if they are better governed, judging by employee satisfaction (Edmans, 2011)
En la realidad y como va avanzando el mundo con respecto a los factores ESG, se ha demostrado lo opuesto a lo anterior mencionado, evidenciando que las empresas verdes superan en rendimiento a largo plazo a los cafés en diversas definiciones y mediciones de estos factores. Tal como lo estudia \citeA{edmans_does_2011}, el cual demuestra que las empresas al tener mejor gobernanza (siendo juzgadas por la satisfacción de sus empleados) suelen rendir mejor que las que no tienen buenas calificaciones en el mismo ámbito. 

De acuerdo a la evaluación del articulo de Edmans donde se evaluaron las mejores 100 empresas para trabajar en Estados Unidos, estas obtuvieron un alfa de cuatro factores de 3,5\% entre los años 1984 y 2009, estando por encima un 2,1\% sobre otras industrias. Además, estas obtuvieron ganancias sorpresas (a que se refiere con sorpresas?) mas positivas y rendimientos de anuncios significativamente mas positivos, siendo un 1,2\% a 1,7\% mas que otras firmas. Lo anterior mencionado se debe a 3 temas fundamentales que se han ido potenciando con el paso de los años; la importancia del capital humano en las empresas, teniendo en cuenta a los empleados como activos esenciales de la organización y no como objetos reemplazables los cuales pueden crear un valor sustancial en la empresa generando nuevas ideas y formando relaciones humanas; lo incapaz que ha sido el mercado para poder entregar toda la información respecto al verdadero valor de los activos intangibles (tal vez profundizar en lo que se refiere intangible), teniendo en cuenta la falta de información de los inversores y otros factores que hacen que la empresa no este realmente bien valorada en el mercado, generando que los inversores no estén conscientes de los verdaderos beneficios de invertir moralmente bien; y el efecto que se esta teniendo al invertir socialmente responsable (SRI en ingles). Con la teoría tradicional de portafolio de Markowitz en 1959 (arreglar citas), la cual indica que poner limitaciones en las decisiones del inversor (tal como sugiere SRI) reduciría los rendimientos, pues una optimización restringida no es mejor que una que no este restringida. Edmans logro relacionar que ciertos filtros  
%Moskowitz en 1972 (arreglar cita), Luck y Pilotte en 1973 y Derwall Guenster, Bauer, y Koedijk en 2005 
de SRI mejoran los rendimientos si se tratan de periodos cortos, siendo la satisfacción del empleado uno de los factores mas colacionados con la rentabilidad de los retornos, dándole así mas valor a las empresas, y por consiguiente superando a empresas cafés.


\vspace{0.5cm}
%Agregar referencias sobre tema 4 
\subsection{Las inversiones socialmente responsables de una empresa aumentan la lealtad de sus clientes}
%(Albuquerque et al. (2019) construyen un modelo en el cual las inversiones socialmente responsables de una empresa aumenten la lealtad de los clientes.)
Lo anterior también es consistente con lo investigado por \citeA{albuquerque_corporate_2019} los cuales lograron construir un modelo en donde se logra reflejar que las inversiones socialmente responsables (ISR), en empresas que invierten en tener una mayor diferenciación en sus productos, permite un aumento en la lealtad de los clientes, dándole así mas poder a las empresas de poder fijar precios, y de esta manera las empresas tienen estadísticamente y económicamente menos riesgo sistemático y mas valor de mercado, para empresas con un alto nivel de ISR. Esto es representado por un análisis con la Q de Tobin, donde se analiza el el efecto de ISR en el valor de la empresa, dando como resultado una relación positiva entre la Q de Tobin y la ISR, siendo mas fuerte con empresas con mayor diferenciación de productos.
\vspace{0.5cm}
%Agregar referencias sobre tema 5 
\subsection{Las inversiones en ASG tienen un impacto social positivo}
%(Oehmke and Opp (2020)La inversión en ESG tiene un impacto social positivo, siento que el paper es un poco alejado a lo de esta investigacion pero igual lo mencione)

De manera significativa, los activos que están bajo gestión en fondos socialmente responsables han ido creciendo con el tiempo, gracias a que los inversores buscan aumentar su asignación en activos con puntuaciones altas de ASG. Esto es por medio de cambios en los gustos que se están realizando en los clientes e inversores, dando como resultado que los activos verdes estén superando a los activos marrones a largo plazo. En consecuencia a esto, el tamaño de la industria en invertir en estos activos ASG, así también como los alfas de los inversores, dependen directamente de que tanto priorizan a las empresas que sigan estas políticas, de tal forma de sacrificar un poco de rendimiento, obteniendo un excedente de inversión al darse el gusto de elegirlas, generando un impacto social positivo. Esto es consistente con lo mencionado por \citeA{202}Oehmke and Opp (2020) en su articulo ``A Theory of Socially Responsible Investment" (ver bien la cita harvard), donde se realiza un estudio caracterizando las condiciones donde un fondo de inversión socialmente responsable conduce a distintas empresas a reducir externalidades, por medio de restricciones de financiamiento y coordinación entre agentes, generando un impacto y producción mas limpia en la inversión. Esto se aleja un poco a lo nuestro, pero enfatiza el impacto social que se genera, por medio de preferencias en tenencias mas verdes.

\vspace{0.5cm}

%Agregar referencias sobre tema 6 
\subsection{La frontera eficiente de ESG}
%(Pedersen (2021) “Responsible investing: The ESG-efficient frontier”)


Con el avance que ha tenido la inversión teniendo en cuenta los factores ASG, han ido existiendo distintas formas de poder valuar los verdaderos beneficios y costos que podría traer, influyendo de esta forma, en la elección de la cartera de activos y sus precios en equilibrio. Es por esto que \citeA{pedersen_responsible_2021}, logro desarrollar una teoría en la cual las puntuaciones ASG de cada activo se dividía en dos roles. Primero en entregar información sobre los principios de la compañía, y segundo, ayudar en las preferencias del inversor. La solución propuesta por Pedersen es una frontera eficiente ASG, que indica el ratio de sharpe mas lógrale según la cartera determinada. Para poder lograr esto se modelaron tres tipos de inversores:  Primero están los que si tienen conciencia ESG (Tipo-A), los cuales maximizan su utilidad, pero tienen un grado de atención a los factores de ASG al momento de tomar decisiones sobre el riesgo y rendimiento utilizado, eligiendo así su cartera con el mayor ratio de sharpe posible. Luego, están los motivados por ASG (Tipo-M), los que utilizan información exclusivamente ASG para formar sus portafolios y buscan con esto formar una cartera a la derecha de cartera tangente donde se encuentra la frontera eficiente ASG, buscando un equilibrio óptimo entre un gran rendimiento del portafolio, bajo riesgo y altos puntaciones de los factores, explicándose en una optimización a las puntaciones elevadas y el ratio de Sharpe. Por ultimo, están los que no tienen conciencia a los factores ASG (Tipo-U) y solo buscan maximizar su utilidad de media-varianza condicional, por lo que pueden elegir una cartera por debajo de la frontera eficiente ASG, pues se basan en menos información sin contar estos criterios.

\vspace{0.5cm}

Tendre que hablar de la regulacion estado unidense que analizaremos aca en la introduccion tambien?
