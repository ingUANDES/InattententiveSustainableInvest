% cap1.tex

\chapter{Introducción}
\label{ch:intro} % la etiqueta para referencias

%{\color{blue} En esta sección de debe incluir el fundamento general de la motivación de la investigación desarrollada en la tesis. También se debe incluir la Hipótesis de trabajo, Objetivo General y específicos. (Máximo 5 páginas). Usar mismo tamaño, tipo de letra y espacio del este formato).}

%\vspace{0.5cm}

%The \Gls{latex} typesetting markup language is specially suitable 
%for documents that include \gls{maths}. \Glspl{formula} are 
%rendered properly an easily once one gets used to the commands.

%Given a set of numbers, there are elementary methods to compute 
%its \acrlong{gcd}, which is abbreviated \acrshort{gcd}. This 
%process is similar to that used for the \acrfull{lcm}.

%\vspace{0.5cm}
\section{Motivación}

Hoy en día se ha observado un creciente interés por parte de los inversores en los criterios Ambientales, Sociales y de Gobernanza (ASG o ESG por sus siglas en inglés).\footnote{En la motivación del Proyecto FONDECYT Drivers of Sustainable Finance se citan datos de que los activos domiciliados en Estados Unidos que utilizan estas estrategias representan un tercio del total de la inversión bajo gestión profesional en US SIF (2020).} Esto se debe en gran medida al aumento en la conciencia de los inversionistas en temas los cuales hacen referencia a desafíos ambientales y sociales, tales como el cambio climático, la escasez de recursos naturales, distintas prácticas laborales y la ética empresarial, insistiendo de esta manera a las empresas que reporten de manera mas constante y transparente sobre estos temas \footnote{En el inciso 2 de 2022 in review: the evolution of the ESG reporting landscape}. A nivel global, la inversión en ASG ha experimentado un crecimiento significativo en la ultima década, llegando a aumentar hasta 10 veces su tamaño en el mercado. Según estimaciones de Morningstar (Empresa estadounidense de servicios financieros), los activos totales en fondos designados a ESG superaron los 3,9 millones de dolares a finales de septiembre 2021. Este crecimiento viene de la mano con el incremento de información que las empresas están dando a los inversores. Además, el numero de empresas cotizada que emiten informes corporativos ha aumentado, pasando de menos de 20 en la década del 1990 a mas de 10.000 en la actualidad. \footnote{The Rise of International ESG Disclosure Standards} Asimismo, es intuitivo pensar que las empresas que logran seguir estas políticas consiguen traer a largo plazo un impacto significativo tanto en la sostenibilidad y rendimiento financiero. Por lo tanto, los tomadores de decisiones buscan elegir estas empresas que consiguen buenos resultados en estos tres factores, y de esta forma evitan compañías que basen sus ganancias en vicios (alcohol, tabaco y juegos). Este tipo de preocupaciones puede mejorar o facilitar condiciones para empresas que desarrollen productos y estrategias de inversión sostenible. Por ejemplo, empezando por fondos de inversión socialmente responsables, bonos verdes o hasta acciones de empresas enfocadas en energías renovables y buenas conductas empresariales. Pero las preguntas a plantearse son, ¿Realmente la gente les toma el real peso a estos criterios? ¿Realmente están atentos a los cambios de regulación que han tenido que implementar las empresas para que de esta forma guíen su decisión de si incorporan o no una empresa sobre la otra en el portafolio de inversión?
\\

En Chile, la importancia de la agenda ASG está en constante crecimiento entre las empresas y sus directivos\footnote{https://accionempresas.cl/content/uploads/esg-en-chile-2023.pdf}, aunque aún hay un largo camino por recorrer. Esto comenzó con las empresas más grandes, que, debido a su significativa presencia en el mercado de valores, están más expuestas a estos asuntos. Estas empresas líderes en el ecosistema empresarial se espera que desempeñen un papel crucial al incentivar a otras empresas a adoptar un enfoque de finanzas sostenibles. El desafío actual radica en que tanto las empresas como sus partes interesadas comprendan y asimilen estos conceptos. Esto implica considerar tanto los aspectos financieros como los no financieros de la actividad empresarial en una misma ecuación, y buscar nuevas formas de alcanzar sus objetivos. Es importante destacar que en la actualidad, no solo se trata de crear valor para los accionistas, sino también para los diversos grupos de interés. Se trata de reconocer que los estándares ASG van más allá de cumplir con regulaciones y figurar en clasificaciones. La agenda ASG brinda la oportunidad de generar un valor adicional para todos y de capturar una parte más significativa de ese valor para la empresa. Además, en un entorno donde los clientes se han vuelto más demandantes, es de creciente relevancia entrar mas de lleno en los estándares ASG, con tal de que las empresas puedan ganarse la preferencia y atención de sus clientes. La adopción de prácticas empresariales adecuadas y socialmente aceptables puede influir en la disposición de los consumidores a adquirir productos o servicios.
De esta misma manera, el día 12 de Noviembre del año 2021, la Comisión para el Mercado Financiero emitió una nueva normativa de carácter general N°461, la cual incorpora nuevos temas a la Memoria Anual de las empresas supervisadas, abordando sostenibilidad y gobierno corporativo, con el fin de poder guiar a las empresas a las empresas a reportes ASG mas robustecidos.
\\

Teniendo en cuenta lo anterior, se busca aplicar un modelo de inversiones sostenibles el cual tenga un cambio en una regulación ASG propuesta por el gobierno, y ver la relevancia de la inatención a estos criterios en la valorización de los Retornos esperados en sus carteras de inversión. Es por esto que se busca por medio de los integrantes mencionados al inicio, una base de datos y un catastro donde se pueda analizar en que se están fijando hoy en día los inversores a la hora de querer invertir en sus portafolios, y si realmente, están al tanto de las regulaciones que cada vez toman más importancia en el ámbito financiero. Además, de querer crear un programa donde se reúna que información es relevante para los analistas al momento de querer crear su portafolio y si toman en cuenta estos factores ASG.

\vspace{0.5cm}

Actualmente la inversión sostenible sigue siendo un área confusa de las finanzas que suele interpretarse de distintas maneras. En la mayoría de los casos, se traduce en la creación de carteras de inversión que excluyen categorías consideradas problemáticas, como la "desinversion" en empresas productoras de combustibles fósiles, en un aparente esfuerzo por abordar el cambio climático. Lamentablemente, existe una diferencia entre abstenerse de participar en algo que uno no desea respaldar y luchar activamente contra algo que cree que debe detenerse por el bien común de todos. La desinversion, que a menudo suele confundirse con los boicots, no tiene un impacto claro en el mundo real, ya que el hecho de que el 10\% del mercado no compre sus acciones no equivale a que el 10\% de sus clientes dejen de adquirir sus productos. \footnote{Articulo creado por Tariq Fancy "The Secret Diary of a ‘Sustainable Investor’"}

\vspace{0.5cm}

Además cada vez hay mas intereses en el impacto financiero que se esta teniendo con estos criterios ASG. Los estudios se centran en evaluar como los parámetros ASG pueden afectar el rendimiento y la rentabilidad de las inversiones. Se estan llevando a cabo investigaciones para determinar si las empresas con mejores practicas ASG tienen un mejor desempeño financiero a largo plazo y si los inversores que consideran estos criterios pueden obtener mejores resultados. En Chile, cada vez se están profundizando mas los criterios ASG, como se puede ver en Santander donde cuentan ya con fondos mutuos ASG, apoyando el medioambiente, la sociedad y las buenas prácticas. Con esto, podemos obtener distintas fuentes y bases de datos con tal de poder trabajar y así investigar de mejor forma en que se están basando en su toma de decisiones los inversionistas.

\vspace{0.5cm}

De acuerdo a la investigación realizada en el artículo de \citeA{pastor_sustainable_2021}, “Sustainable investing in equilibrium”, donde se modelan las inversiones realizadas en el mercado teniendo en cuenta los factores ASG (Ambiental, sociales, gobernanza) de las empresas, se demostró que, en equilibrio, los activos verdes (que tienen buenos factores ASG) superan el rendimiento esperado de los activos cafés (no con buenos factores ASG) cuando se producen impactos positivos en el factor ASG, el cual refleja cambios en los gustos de los consumidores por productos ecológicos y en los gustos de los inversores por tenencias verdes. También se hace mención de un modelo el cual de alguna forma calcula el gusto de la gente de querer añadir inversiones “verdes” a sus portafolios con tal de que ha futuro se generen retornos esperados futuros, con el plus de además de no dañar el medio ambiente, u otros factores ASG. 

\vspace{0.5cm}



\section{Hipótesis}

Esta tesis enuncia 2 hipótesis de trabajo descritas a continuación:

\begin{enumerate}
    \item Un cambio regulatorio en los estándares de reporte de métricas de cumplimiento relacionadas a ESG tiene efecto en la valorización de activos financieros. 
    \item Dada la complejidad de las métricas ESG, existe un componente en la valorización de activos relacionado a inatención racional.
\end{enumerate}

\section{Objetivos}

El trabajo se estructura en torno a un objetivo general que articula las hipótesis y N objetivos especificos que ...

\subsection{General} 

Evaluar el rol de la inatención racional de los tomadores de decisión ante un cambio en la regulación sobre reporte de métricas ESG en un modelo de inversión sostenible con varios activos en equilibrio.

\subsection{Específicos} 

Tomando en cuenta la revisión de literatura de \cite{pastor_sustainable_2021} y el modelo propuesto, estructuramos los siguientes objetivos específicos.

\begin{enumerate}
    \item Contextualizar en base a la literatura relacionada considerando los temas de: Compromiso de los inversionistas, subrendimiento de activos sostenibles, sobrerendimiento de activos sostenibles, relación entre inversión sostenible y lealtad cliente-empresa, impacto social de la inversión sostenible, y la caracterización de una frontera eficiente.
    \item Identificar resultados del modelo de \cite{pastor_sustainable_2021} que puedan ser validados con datos
    \item Incorporar inatención racional en la toma de decisiones del inversionista del modelo de \cite{pastor_sustainable_2021}
    \item Construir una base de datos para validar el objetivo específico 2 en dos periodos de tiempo (antes y después del cambio en la regulación)
    \item Analizar el grado de validación de las hipótesis de trabajo
\end{enumerate}


\section{Estructura del informe}

En este capítulo \ref{ch:intro} revisamos.... En el próximo Capítulo \ref{ch:litRev} veremos... En el Capítulo \ref{ch:metodo} ...

{\color{blue} (2 a 3 líneas que informe al lector el contenido de cada capítulo) }